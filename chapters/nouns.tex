\chapter{Modeling Resources}

As we dive into modeling our API for our first user stories, we'll be using Open API 3.  You can find the latest resources for Open API 3 at \url{https://github.com/OAI/OpenAPI-Specification}.  Take note that there is both a community and a business behind this modeling specification, \url{https://smartbear.com/}, and they are deserving of support.  The fastest way to begin modeling your API is by using their resources at \url{https://editor.swagger.io/}.

When modeling a RESTful object state, pay careful consideration to the state being akin to a noun.  You should not have a component definition of GetSearchResult, it would just be SearchResult.

\begin{sidebar}
\begin{center}
\textbf{Object State, Component, or Resource?}
\end{center}

Our modeling markdown choice of Open API 3.0 uses the label "component", REST labels it "object state", and URL is an acronym for uniform "resource" locator.  \textit{We will use component, resource, and object state interchangeably when discussing modeling the RESTful object.}

\end{sidebar}

\section{Modeling REST Object State}

User Story:

\textit{I as a Borrower need check on a book's availability so that I may know if the book I wish to checkout is available.}

Based on this user story, we know the resource we're looking to retrieve is a book.  From speaking with our product owner, and reviewing our user stories, we can tell we are speaking about a book copy.  We can also speak to our product owner regarding restrictions for this search.  Must the user be authenticated, or can an anonymous user inspect the availability for a certain title?  In this case, we're going to allow any unauthenticated user to research a book's availability.

Let's start with some assumptions based on fictitious conversations with our product owner.  For our book search story, to be discussed later, we know that we want our books to be identified by title, authors, copyright date, and International Standard Book Number (ISBN-13).

\begin{minipage}{\linewidth}
We can start by modeling the applicable resources using Open API 3.0 YML.  This component YML can be placed within the editor at \url{https://editor.swagger.io} and you can review how the resulting object will look.
\begin{code}
\begin{lstlisting}[belowskip=-\baselineskip]
components:
  schemas:
    Book:
      type: object
      properties:
        isbn:
          type: string
          description: International Standard Book Number
          example: 978-1119096726
        title:
          type: string
          description: Book title
          example: Applied Cryptography
        authors:
          type: array
          items:
            type: string
          example: ["Bruce Schneier"]
        copyrightDate:
          type: integer
          example: 2015
\end{lstlisting}
\end{code}
\end{minipage}

\begin{sidebar}
\begin{center}
\textbf{JSON CamelCase, Dashes, or Underscore?}
\end{center}

It is left up to the API designer to choose between camelCase, kebab-case (dashes), or underscores for separators.  That said, Google's JSON Style Guide at \url{https://google.github.io/styleguide/jsoncstyleguide.xml?showone=Property_Name_Format#Property_Name_Format} standardizes on camelCase.  We will follow camelCase for our JSON definitions.

\end{sidebar}

\begin{sidebar}
\begin{center}
\textbf{Yet Another Markup Language}
\end{center}

YAML, pronounced yam-el, is short for either "Yet Another Markup Language" or the recursive "YAML Ain't Markup Language" depending on who you ask.  It is becoming more and more common for specification and configuration files.  Golang, Python, Java, and others have parsers for it.  It is the most common configuration format for Spring Boot, now emitted by the Spring Boot initializers by default.  At its core it has the look and feel of Python because space matters, but it is a very clean, readable, and concise markup for configuration and now Open API 3.0 specification.  Expect to see APIs, potentially even REST APIs, offered with YML as the reply format.

\end{sidebar}

So, with our book resource defined, we know we need to add a list of available copies, but when we go to edit a book, are we editing a book or a copy the library has?  If you were paying very close attention, you likely noticed that we have a book resource, and a copy resource.  They are not the same thing.  This is the first curveball of our REST definition, of which we will encounter many.  Simple user stories lead to quite a bit of complexity in API design, but with some careful planning, you can navigate and make your team of engineers' lives much easier.

\begin{minipage}{\linewidth}
\begin{code}
\begin{lstlisting}[belowskip=-\baselineskip]
components:
  schemas:
    BookCopy:
      type: object
      properties:
        isbn:
          type: string
          description: International Standard Book Number
          example: 978-1119096726
        barcode:
          type: string
          description: Copy identifier used for scanning and unique text copy result.
\end{lstlisting}
\end{code}
\end{minipage}

OK, so we're still at an awkward place in terms of defining number of available copies.  Would we include that on the book?  Well that's not the right place.  Would we include that on a copy?  Well no, we expect a book copy to be singular.  What we're uncovering is that we have a read-only object to define, a resulting resource from a search.  You will commonly find that search results require a defined component which feels out of place because it won't map into your back-end system's object storage.  This is where the back-end focused architects may struggle and where consumer-driven-contracts becomes key.  We care about our consumers' needs first.  That said, our result needs to be a search result.

\begin{minipage}{\linewidth}
\begin{code}
\begin{lstlisting}[belowskip=-\baselineskip]
components:
  schemas:
    BookAvailability:
      type: object
      properties:
        isbn:
          type: string
          description: International Standard Book Number
          example: 978-1119096726
        available:
          type: integer
          minimum: 0
          description: Available copies for checkout or reservation, ranging from 0 to N
        checkedOut:
          type: integer
          minimum: 0
          description: Total copies checked out
\end{lstlisting}
\end{code}
\end{minipage}

\begin{sidebar}
\begin{center}
\textbf{Art Over Science}
\end{center}

Looking at this BookAvailability definition for a search result, we would have a debate over if the result should include ISBN.  Let's suppose that we are planning to provide an endpoint to find available copies by ISBN, why would we echo the ISBN back?  If there's a reason, why not just echo the entire Book component back and include the title, the author, etc.

This is where a choice of style comes into play, and engineers will disagree.  My opinion is that when our browser-based clients are constructing asynchronous JavaScript processing results in promise blocks, debugging will be much easier for them if there is a reference between search result and search text.  Should that be the full Book component?  Perhaps, but you can also make the case that the Book will likely grow and grow over time and feature requests, say maybe include pages, publisher information, secondary publishing details, etc.

There are merits to include the ISBN, not include the ISBN, and provide a full embedded result of the Book that was used for searching available copies.  All three are right, and all three have reasons to be wrong, and that tells us that some of our decisions here are going to be Artistic and not Scientific.
\end{sidebar}
