\section{Consistency is Key}

Be warned that when planning out your API's context, most of what you're doing isn't unique, or at least it shouldn't be. Chances are the context headers you're needing have been standardized and documented.  Leverage web searches to find common header HTTP header labels and HTTP status codes.  More importantly, if you're leveraging 3rd party API Gateways and network topology toolkits, they're looking for common headers, user agents, correlation headers, and more.  Do not reinvent the wheel when it comes to context headers.  

If you're looking for an appropriate status code for your client's failure, say from providing an expired authentication, that's going to be a 401 and you're quickly led to that conclusion with a quick search.  You can find a quick run-down of common status codes at \url{https://developer.mozilla.org/en-US/docs/Web/HTTP/Status}.

For tracing and correlation of requests and client session activity, it has again been done before.  See \url{https://en.wikipedia.org/wiki/List_of_HTTP_header_fields} which will lead you to \textit{X-Request-ID} and \textit{X-Correlation-ID}.

The point is that if what you're designing is primarily contextual or technical, not related to your API's custom object state, try to stick to common headers and status codes.  Your consumers will thank you for it.
