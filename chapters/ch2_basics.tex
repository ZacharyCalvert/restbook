\chapter{Basics of Modeling REST}

As we dive into modeling the API for our first user stories, we'll be using Open API 3.  You can find the latest resources for Open API 3 at \url{https://github.com/OAI/OpenAPI-Specification}.  Take note that there is both a community and a business behind this modeling specification, \url{https://smartbear.com/}, and they are deserving of support.  The fastest way to begin modeling your API is by using their resources at \url{https://editor.swagger.io/}.

\begin{sidebar}
\begin{center}
\textbf{Open API 3}
\end{center}

We will leverage Open API 3 specification to model our REST API, but this is not a book diving into the full specification of it.  You can read the full specification in an hour and it is worth doing so after you've ready this book.  Our goal here is to present a model of REST endpoints without selecting a programming language to implement it in.  There are a few leaders in the modeling options including RAML, Open API 3, Swagger 2, and API Blueprint.  Visit \url{https://github.com/OAI/OpenAPI-Specification} for the full specification.

\end{sidebar}

When modeling a RESTful object state, pay careful consideration to the object state being akin to naming and describing a noun.  You should not have a component definition of GetSearchResult, it would just be SearchResult.

\begin{sidebar}
\begin{center}
\textbf{Object State, Component, or Resource?}
\end{center}

Our modeling markdown choice of Open API 3.0 uses the label "component", REST labels it "state", and URL is an acronym for Uniform \underline{Resource} Locator.  \textit{We will use component, resource, and object state interchangeably when discussing modeling the RESTful object's properties.}

\end{sidebar}

\section{Modeling REST Object State}

User Stories we will start with for resource modeling:

\begin{itemize}
  \item I as a Borrower need to check on a book's availability so that I may know if the book I wish to checkout is available.
  \item I as a Borrower need to search for a book by title so that I may know if the book I wish to checkout is available.
  \item I as an Admin need to reserve a copy of a book for a patron so that they may know it is available when they arrive.
\end{itemize}

Based on these three user stories, we can go through and determine the resources we need to model.  We know we have a Book, or really a book's metadata.  We also know we have to model the physical copy of a text, such as a uniquely scannable barcode for a library to check-in and check-out.  We can then speak to our product owner regarding restrictions for the availability search.  Must the user be authenticated, or can an anonymous user inspect the availability for a certain title?  In this case, we're going to allow any unauthenticated user to research a book's availability.

Let's start with some assumptions based on fictitious conversations with our product owner.  For our book, we know that we want to include title, authors, copyright date, and International Standard Book Number (ISBN-13).

\begin{minipage}{\linewidth}
We can start by modeling the applicable resources using Open API 3.0 YML.  This component YML can be placed within the editor at \url{https://editor.swagger.io} and you can review how the resulting object will look.
\begin{code}
\begin{lstlisting}[belowskip=-\baselineskip]
components:
  schemas:
    Book:
      type: object
      properties:
        isbn:
          type: string
          description: International Standard Book Number
          example: 978-1119096726
        title:
          type: string
          description: Book title
          example: Applied Cryptography
        authors:
          type: array
          items:
            type: string
          example: ["Bruce Schneier"]
        copyrightDate:
          type: integer
          example: 2015
\end{lstlisting}
\end{code}
\end{minipage}

\begin{sidebar}
\begin{center}
\textbf{JSON CamelCase, Dashes, or Underscore?}
\end{center}

It is left up to the API designer to choose between camelCase, kebab-case (dashes), or underscores for separators.  That said, Google's JSON Style Guide at \url{https://google.github.io/styleguide/jsoncstyleguide.xml?showone=Property_Name_Format#Property_Name_Format} standardizes on camelCase.  We will follow camelCase for our JSON definitions.

\end{sidebar}

\begin{sidebar}
\begin{center}
\textbf{Yet Another Markup Language}
\end{center}

YAML, pronounced yam-el, is short for either "Yet Another Markup Language" or the recursive "YAML Ain't Markup Language" depending on who you ask.  It is becoming more and more common for specification and configuration files.  Golang, Python, Java, and others have parsers for it.  It is the most common configuration format for Spring Boot, now emitted by the Spring Boot initializers by default.  At its core it has the look and feel of Python because space matters, but it is a very clean, readable, and concise markup for configuration and now Open API 3.0 specification.

\end{sidebar}

So, with our book resource defined, we know we need to add a list of available copies.  So, why can't we just put a list of copies on the Book resource itself?  The argument can even be made in the is-a vs has-a relationship OOP model that a copy of a book is-a book, so from a polymorphic language a book copy could just inherit from Book.  You would be correct.  However, when we're modeling our REST endpoint URLs, searching for books by title vs retrieving available copies by ISBN will have different URL paths.  It would be completely acceptable to start modeling resources, then get to the endpoints, and come to the conclusion that when modeling your endpoints you want to split up the object instead of treating it as an inherited object.  An Object-Oriented Programming (OOP) purist would even struggle with the Book labeling, and show that it should really be something along the lines of BookMeta, or BookDescriptors.

\begin{minipage}{\linewidth}
\begin{code}
\begin{lstlisting}[belowskip=-\baselineskip]
components:
  schemas:
    BookCopy:
      type: object
      properties:
        isbn:
          type: string
          description: International Standard Book Number
          example: 978-1119096726
        barcode:
          type: string
          description: Copy identifier used for scanning and unique text copy result.
\end{lstlisting}
\end{code}
\end{minipage}

By including the ISBN in our BookCopy, we're violating data duplication tenets of OOP.  Again the question comes up, does a BookCopy have a Book?  Technically, we're not in OOP here.  When we're modeling object state for REST, we want lean results, driven by consumer needs.  On the one hand, OOP would decry this approach.  In our case, we want to be mindful of the fact that the Book object may grow and grow, impacting network costs when returning say an array of BookCopy results when providing an endpoint for all of the available copies assigned to a library.  Imagine the Library of Congress, which could have 50 copies of \textit{Catcher in the Rye}.  If I was to request all BookCopies for \textit{Catcher in the Rye}, I would not want the full Book object embedded within every element of the array.  The reality is, this is a bit of a balancing act.  You could also strip away the referential ISBN between BookCopy and Book, but then you would have no means to scan a book's barcode, look up the copy by barcode, and relate it to a Book.

That feeling of "that's not right", is because we're breaking typical OOP, and there are no hardened rules here.  Networking costs, balancing referential data between objects, staying focused on client user stories, it is a bit of a complex burden.  For such a simple, ancient problem of modeling and architecting a library's set of RESTful endpoints, there's a lot to consider and a lot to model.

All of that said, we're still at an awkward place in terms of defining number of available copies.  Would we include available count on the Book?  Well that's not the right place because we've chosent to split up Book and BookCopy.  Would we include available count on a BookCopy?  Well no, we expect a book copy to be singular.  What we're uncovering is that we have a read-only object to define, a resulting resource from a search.  You will commonly find that search results require a defined component which feels out of place because it won't map into your back-end system's object storage.  This is where the back-end focused architects may struggle and where consumer-driven-contracts becomes key.  We care about our consumers' needs first.  That said, our result needs to be a search result.

\begin{minipage}{\linewidth}
\begin{code}
\begin{lstlisting}[belowskip=-\baselineskip]
components:
  schemas:
    BookAvailability:
      type: object
      properties:
        isbn:
          type: string
          description: International Standard Book Number
          example: 978-1119096726
        available:
          type: integer
          minimum: 0
          description: Available copies for checkout or reservation, ranging from 0 to N
        checkedOut:
          type: integer
          minimum: 0
          description: Total copies checked out
\end{lstlisting}
\end{code}
\end{minipage}

Looking at this BookAvailability definition for a search result, we would have a debate over if the result should include ISBN.  Let's suppose that we are planning to provide an endpoint to find available copies by ISBN, why would we echo the ISBN back?  If there's a reason, why not just echo the entire Book component back and include the title, the author, etc.

Again, this is where a choice of style comes into play, and engineers will disagree.  My opinion is that when our browser-based clients are constructing asynchronous JavaScript processing results in promise blocks, debugging will be much easier for them if there is a reference between search result and search text.  Developer tools showcasing the network request/response data will also be easier to trace when including relational data.  Should the relational data be the full Book component?  Perhaps, but you can also make the case that the Book will likely grow and grow over time and feature requests, say maybe include pages, publisher information, secondary publishing details, etc.

There are merits to include the ISBN, not include the ISBN, and provide a full embedded result of the Book that was used for searching available copies.  All three are right, and all three have reasons to be wrong, and that tells us that some of our decisions here are going to be Artistic and not Scientific.

If these modeling decisions were left to a committee of architects, I'd anticipate quite a bit of argument here.  The joys of having too many cooks in the kitchen.

\emph{Object Modeling Summary:}

\begin{itemize}
  \item Use Consumer-Driven Contracts over Object Oriented Programming to design your objects.  We're not in OOP anymore.
  \item Designing RESTful object state is as much art as science.
  \item Object state should be labeled as a resulting noun (person, place, or thing) and not be prefixed by a verb for the interaction.  Use Book instead of GetBook.
\end{itemize}

\section{HTTP Methods and REST}

When speaking to or as a system architect, HTTP Methods are known as REST Verbs.  It is the action you are taking against the resource.  You're posting it, getting it, putting it, or deleting it, so the terminology is appropriate.  HTTP Method and REST Verbs are interchangeably used in REST API architecture conversations.

The most common RESTful verbs for mapping out create-read-update-delete (CRUD) operations are GET, PUT, POST, and DELETE.  There is a lesser known PATCH operation used as well, which serves as a sparse data update.  Finally, you will encounter OPTIONS when using REST services from a browser, particularly when going across domains, say from www.example.com hitting an API offered using a subdomain such as rest-api.example.com.

To provide a quick preview, HTTP GET is object retrieval or collection search, and shouldn't incur any side effects.  It is imperative, from a security perspective even, that HTTP GET methods inflict no change.  Why?  The fastest answer to give would be that some services still leverage long-lived session cookies for user identity.  Imagine I am a customer at a bank, and they have my authenticated session stored as a cookie in my browser, and I happen upon an emailed link which works along the lines of \textit{/transfer/to?account=hackers\&amount=500.00} which says "click to see video here" as the presented text in the email.  Bad day.  Just don't incur changes to object state as a GET.

HTTP POST is meant for resource creation.  HTTP PUT is for resource update \textit{or resource creation}.  HTTP DELETE is for deletions or cancellation.  Finally, HTTP PATCH should be used for a sparsely filled updates against large resources, or content that is easier for the client to update portions of the resource instead of the whole resource.  Expect a much more thorough explanation in the coming chapters.

\begin{minipage}{\linewidth}
\begin{sidebar}
\begin{center}
\textbf{Security First}
\end{center}

The security aspects of designing a RESTful API are presented within this book after presenting the foundational concepts.  This is not the correct approach to take when designing an API.  Security must be a first class citizen when preparing RESTful services.  Whether secured via an API gateway enforcing access controls or building a zero-trust network of microservices, know your security approach before you start implementing.  Retrofitting security into insecure APIs is a mistake and one that could cost your company dearly.  Your clients will need to know your identity provider, you will need to know how to validate the principal, you will need to know how to determine level of authorization, and you will need to have a plan for non-prod test accounts and test credential automation for integration testing.

\end{sidebar}
\end{minipage}

\emph{Method Summary:}

\begin{itemize}
  \item HTTP Methods GET, PUT, POST, DELETE, and PATCH are also known as REST verbs.
  \item HTTP GET methods are read-only must incur no side effects to the object state.
  \item HTTP POST is create.
  \item HTTP PUT is update or create.
  \item HTTP DELETE is deletion or cancellation.
  \item HTTP PATCH is sparse update.
\end{itemize}

\section{Modeling Endpoints}

When designing your REST endpoint, start by identifying the nouns which we can describe state for.  In our case study, we have books and book copies.  In a much more complete application plan, we'd also have users, libraries, publishers, and potentially more.  Start by determining the object states your consumers are going to need to retrieve, and then model that as a collection.  For example, \textit{/books} and \textit{/copies}.  When describing a collection of resources such as books or copies, use the plural in the URI and treat the identifier as a path component.  For example, \textit{/books/book-id} and \textit{/copies/copy-barcode}.

One thing a good system architect will always be ready for is change.  We plan for it because we know it will inevitably happen, either from design mistakes, technology updates, changes in feature requirements, new feature additions, etc.  Designing REST can set the level of preparedness for the inevitability of change.  We will version our endpoints, assuming that contract breaking changes will happen over the coming releases.  A hot topic will be do we place the version in the endpoint URL, or do we make use of HTTP Headers, or do we make use of subdomains.

Three options commonly available:

\begin{itemize}
  \item http://www.example.com/v2/books/book-id
  \item http://v2.example.com/books/book-id
  \item http://www.example.com/books/book-id + HTTP Header such as X-Version: 2.0
\end{itemize}

My recommendation will be that you place the version in the URL, so \textit{/v2/books/book-id}.  The benefits include:
\begin{itemize}
  \item Our site access logs will include both the version of the API and the HTTP status code, so we can quickly tell if we're getting HTTP status errors from an old version or a newer one, particularly during updates to our API suite.
  \item We can document multiple versions within the same Open API 3.0 contract document.
  \item An API Gateway can easily route traffic based on URL.
  \item Common REST frameworks, such as Spring Boot, can isolate controllers by version of the API.  This will allow one application to process both contract versions.
\end{itemize}

Our choice of placing the version into the URL is Art over Science, as all three flavors of versioning are viable.  The most important takeaway is that you plan for change and have versioning prepared in your API contract at the very first iteration.  I warn that there are architects who will disagree with the recommendation to include version in the URL and expect the version to be an HTTP header because the version is contextual; they're right.  There are others that will want the version to be placed in the domain, so they can manage routing the request by domain; they too are right.

\section{Consumer Driven Contracts}

A core API design tenet to touch upon before we start designing our API is Consumer Driven Contracts, which we will use throughout our design process.  Presenting theory for the pattern for Consumer Driven Contract is quite simple:  \textit{the consumer's needs drive the contract of the resulting API.}

The idea of Consumer Driven Contracts is not new to REST API engineers.  Many SOAP and other TCP services are built with the consumer in mind, but it is worth revisiting for architects planning out RESTful APIs.  At its core, the concept of Consumer Driven Contracts places the burden on the back-end engineers to write the service once, so that the N number of client integrators don't have to rewrite the business logic to accomplish the same task.  The ideal thin client is facilitated by a clean API answering its requirements sufficiently and cleanly.

While the architect for APIs tends to be experienced back-end systems engineers, there is quite a bit of merit to assigning the front-end engineers as the designers. Consumer-Driven Contracts places the burden on the back-end teams to develop systems which hide the complexity through a consumer friendly API. We will hide all of the legacy, all of the technical debt, and all of the poor technology choices behind a clean API meant to enable clients to quickly integrate to.

Our goal for clients is fast adoption so that we can build native and browser apps for Android, iOS, desktop, and command-line consumers.  In addition, thought is given to scenarios where your consumers may be other microservice apps.

When designing an API, it is imperative you have a notion of your consumers, be them microservices, native clients, or terminal system admins.  More importantly, even if you're the developer working on the back-end providing the API offering, you must know the user stories and use cases driving the need for your implementation. Take for example the following user story for a library:

\textit{I as a library administrator need to determine the availability of a book by ISBN so that I may answer availability questions regarding a title.}

This user story could lead to a very different API offering than:

\textit{I as a system need to obtain the list of book copies by ISBN so that I may determine if too many books have been purchased and should be offered for sale.}

The librarian needs to know what's available right now, but the system may need how many are checked out and how many are owned by the library in total.  You can answer both needs with the same REST endpoint, but notice how the system story expands the scope of the resulting API to include copy counts.  Proper planning given into the final API may result in reduced work performed by the system teams to cover the breadth of the user stories.

The argument can be made that this approach smells like a waterfall planning model verses feature planning for an Agile sprint, and you'd be correct.  That said, an architect with a few discussions with a product owner can plan out a strong API in a single sprint, without having to scope the entire project out in a waterfall model.  That doesn't mean the API won't change, and that doesn't mean the API will be perfect, but the planning can pay off in big ways.

\section{Consistency is Key}

Chances are, when planning out your API's context that most of what you're doing isn't unique.  It has been done before and chances are it has been standardized and documented.  Leverage web searches to find common header HTTP header labels and HTTP status codes.

If you're looking for an appropriate status code for your client's failure, say from providing an expired authentication, that's going to be a 401 and you're quickly led to that conclusion with a quick search.  You can find a quick run-down of common status codes at \url{https://developer.mozilla.org/en-US/docs/Web/HTTP/Status}.

For tracing and correlation of requests and client session activity, it has again been done before.  See \url{https://en.wikipedia.org/wiki/List_of_HTTP_header_fields} which will lead you to \textit{X-Request-ID} and \textit{X-Correlation-ID}.

The point is that if what you're designing is primarily contextual or technical, not related to your API's custom object state, try to stick to common headers and status codes.  Your consumers will thank you for it.

