\chapter{REST Verbs}

As we move into describing REST verbs, we will make some assumptions that context information will be

\section{REST GET}

Moving on to our first modeled endpoint, let's review the user story.

\textit{I as a Borrower need check on a book's availability so that I may know if the book I wish to checkout is available.}

From this story, we can tell we want an endpoint to retrieve our BookAvailability resource.  Reaching out to our product owner, we query whether or not we would assume this shows up as part of a book search by title, or if we can assume they already have the unique ISBN.  For our purpose, we will assume a search has been a performed, the user selected from a list of matching titles, and now the client has the ISBN.  What we're looking to design here is the URL for book copy availability by ISBN.

When modeling URLs for REST, there are many considerations to make.  First, is there one resource or multiple resources of the same type?  If the resource is plural, the URL should include a plural, such as /v1/books, or /v1/copies.  Second, is the operation uniquely identified by an identifier, meaning a result of one, or is it retrieval of the list of resources?  If a filtered list with potentially multiple non-unique results, use query parameters such as \url{/v1/books?title=REST}.  If we are looking to identify a unique book, we could model \textit{/v1/books/\{isbn\}} as our URL.

Probably the trickiest portion of modeling a REST API's URLs is relative pathing for relational resources.  Take for example our BookCopy resource being a copy of a Book.  We could potentially model our URLs as \textit{/v1/books/\{isbn\}} and \textit{/v1/copies/\{isbn\}}.  We can also consider the relational aspect of a copy to a book and make copies a sub resource of books, such as \textit{/v1/books/\{isbn\}/copies}.  A guideline here is, \textit{choose the URL which is easier for the consumer to understand}.  If you took the \textit{/v1/copies/\{isbn\}} by itself, it isn't very expressive regarding a copy of what.  However, \textit{/v1/books/\{isbn\}/copies} is much more expressive indicating we're looking at copies of a book.

\begin{minipage}{\linewidth}
Choosing the more expressive endpoint, we can now model our happy-path book availability URL:

\begin{code}
\begin{lstlisting}[belowskip=-\baselineskip]
paths:
  /v1/books/{isbn}/copies:
    get:
      summary: Retrieve available copies of a given Book
      operationId: retrieveAvailability
      parameters:
      - name: isbn
        in: path
        description: ISBN-13 book identifier
        required: true
        schema:
          type: string
      responses:
        200:
          description: ISBN found, server able to identify available copies of book identified by ISBN.
          content:
            application/json:
              schema:
                $ref: '#/components/schemas/BookAvailability'
\end{lstlisting}
\end{code}
\end{minipage}

Now we've modeled our object state as well as URL for checking on availability.  Let's move on to our search story.

\textit{I as a Borrower need to search for a book by title so that I may know if the book I wish to checkout is available.}

Given we've already defined an enpoint of \textit{/v1/books/\{isbn\}/copies}, we can see we're not looking to search for copies, we're looking to search on books.  Our endpoint for REST modeling here would be attached to \textit{/v1/books}.  We're looking to model a search capability for our Borrower actor, and we know they don't know the ISBN off of the top of their head.  When you can't uniquely identify the resource by ID and your case requires filtering or searching, use query parameters.   In this case, we will model the endpoint with a title query parameter.

\begin{minipage}{\linewidth}

\begin{code}
\begin{lstlisting}[belowskip=-\baselineskip]
/v1/books:
  get:
    summary: Search for books
    operationId: retrieveBooks
    parameters:
    - name: title
      in: query
      description: Title match for book search
      required: true
      schema:
        type: string
    responses:
      200:
        description: Matching titles found
        content:
          application/json:
            schema:
              type: array
              items:
                $ref: '#/components/schemas/Book'
\end{lstlisting}
\end{code}
\end{minipage}

Reviewing our resource, we can question how many matches could we afford before it becomes a problem.  What happens if our user searches for "The"?  Coordinating with product owner, our library search feature is problematic because we can't answer with the entire library full of books, and the UI direction is to allow for drop-down matching when additional letters are typed.  Again, leaning on a fake conversation, let's generate a paged result for the search result.

\begin{minipage}{\linewidth}

\begin{code}
\begin{lstlisting}[belowskip=-\baselineskip]
BookSearchResultPage:
  type: object
  properties:
    totalMaches:
      type: integer
      description: Total number of matched books
    offset:
      type: integer
      description: Match offset for pagination
    books:
      type: array
      items:
        $ref: '#/components/schemas/Book'
      description: Matching books
\end{lstlisting}
\end{code}
\end{minipage}

And to enable our consumers to make use of the paging result, we'll update our search endpoint:

\begin{minipage}{\linewidth}
\begin{code}
\begin{lstlisting}[belowskip=-\baselineskip]
/v1/books:
  get:
    summary: Search for books
    operationId: retrieveBooks
    parameters:
    - name: title
      in: query
      description: Title match for book search
      required: true
      schema:
        type: string
    - name: offset
      in: query
      description: Match offset for retrieving next page of matching titles
      required: false
      schema:
        type: integer
    responses:
      200:
        description: Matching titles found
        content:
          application/json:
            schema:
              $ref: '#/components/schemas/BookSearchResultPage'
\end{lstlisting}
\end{code}
\end{minipage}

\emph{HTTP GET Summary:}

\begin{itemize}
  \item HTTP GET is for read only services and must not trigger object state change (side effects)
  \item Model your resource URLs such that they are relative in the path when related, such as our \textit{/v1/books/\{isbn\}/copies}
  \item When uniquely identifying a resource, the ID belongs on the URL path.  When filtering or searching for a matching set of resources, use query parameters.
\end{itemize}
