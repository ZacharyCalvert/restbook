\chapter*{Preface}
\addcontentsline{toc}{chapter}{Preface}

Welcome to Designing, Modeling, and Securing RESTful APIs.  Thank you for your interest and thank you for your support.

This book is written for software engineers and architects who have development experience with a modern language or two and hopefully have been responsible for a nontrivial integration.  You might have even encountered an API which "felt off" or was hard to understand, and couldn't quite identify what was wrong with it.  Encountering a poorly designed API can make you question the quality of the offered product, and is analogous to a bad first date.  A good deal of the time, a first impression really can tell you all you need to know.


For the sake of brevity, this is not really a book for software developers just getting started in their career.  Pairs programming and proper mentorship are a fundamental starting point for someone responsible for designing public facing services.  At a minimum, you should have a microservice or two under your belt.  I learned more on my job in the first 3 months in a professional setting than I did in my 4.5 years of college.

The structure of the book is to start with an explanation of a simple HTTP over TCP exchange, a brief history, and a case study which we will follow throughout the book.  Security, while a significant portion of our focus, is presented after the foundational concepts of REST.  Architectural concepts are presented with thought given to security, scalability, and maintainability. Debugging and diagnosing integration woes is peppered throughout the texts within.

Throughout this book, we will introduce development utilities, debuggers, proxies, and command line utilities.  You'll want to have access to a BASH or ZSH terminal for access to curl and tcpdump.  As expected, you'll also want access to a modern browser with development tools; in 2020 just about any widely available browser will do.  You'll need access to the internet for reviewing the current Open API specification and online editors such as <uri>https://editor.swagger.io/</uri>.  You'll likely also need desktop applications for packet capture reviewing such as WireShark.  Plan to also install developer applications such as POSTMAN and Charles Proxy.  Finally, I try to use containers for all of my demonstration materials, so you may as well go ahead and install Docker command line utilities as well.

Utilities, examples, and sample code used to prepare reference materials for this book have been executed on a Windows 10 machine, running an Ubuntu Windows Subsystem for Linux (WSL).  Frankly I prefer Mac, but my Lenovo has been tried and true, at about a third of the cost of the equivalent MacBook Pro.

This book's initial outline and first draft has been written using DocBook XSL.  I'm not overly fond of the format, and the community and tooling seems very fragmented, but it has allowed me to work within the confines of an Atom text editor and a terminal.  Oddly enough, for a framework on writing documentation, the documentation for DocBook is nearly impossible to wade through, with broken version patterns, impossible to find reference XML, an old-school affinity for manually customized XSLT files, and ancient references to content hosted on the defunct and virus-ware download button laden SourceForge.net.

A fitting question for a preface is "why another book on REST"?  Ultimately, this book will share a shelf with many others written on the same concept, a 20+ year old topic introduced by Roy Fielding in 2000 <citation>Fielding2000</citation>.  I picked up a few of the higher rated REST books to figure out if it was worth the effort and can say confidently this book will fill a niche. Most REST API books are either do's or don't do's, guidelines, standards, or opinion pieces on which approach is better.  I have not encountered one that transitions from a case study into a how-to.  I have yet to uncover one that goes into depths of debugging, CORS, Basic Auth, Bearer Auth, and TLS under the same cover.  Finally, I have come across none using Open API 3.0 spec as their modeling language.  All of that said, most of my knowledge has come from encountering both good and bad APIs, a lot of Stack Overflow posts, a thorough reading of the Open API 3.0 specification, and paid professional enterprise application development experience.

\textcolor{blue}{This is a sample text in blue.}
\colorbox{Gainsboro}{sdfASDF} \medskip
